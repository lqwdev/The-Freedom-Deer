\documentclass[10pt]{article}
\usepackage[document]{ragged2e}
\usepackage{multicol}
\usepackage[margin=1in]{geometry}
\usepackage{titlesec}
\usepackage{fancyhdr}

\pagestyle{fancy}
\fancyhf{}
\fancyfoot[R]{P. \thepage}
\fancypagestyle{plain}{
    \renewcommand{\headrulewidth}{0pt}
    \fancyhf{}
    \fancyfoot[R]{P. \thepage}
}

\setlength{\parindent}{0em}
\setlength{\parskip}{1em}
\titlespacing*{\section}
{0pt}{0.5em}{0.5em}

\title{Project Proposal}
\author{The Freedom Deer: Tianchang Li, Hualiang Qin, Qingwei Lan}

\begin{document}
\maketitle


\section{Theme}

The theme of our research revolves around answering the following guiding questions:
\begin{itemize}
\setlength\itemsep{0em}
\item What makes someone more risky towards experiencing the use of force?
\item What are the typical scenarios in which police officers tend to use force?
\end{itemize}

Our main goal is to investigate the conditions under which police officers tend to use force on civilians. We believe that there are many reasons including the race of the subject, the environment under which the conflicts happened, the physical location in Chicago (neighborhood demographics), etc. We would like to explore how each of these conditions influences the police officer’s choice to use force and how they interact with each other to impact as a whole.

Specifically, we revolve our research on the differentiated treatments civilians receive from the police officers by studying the characteristics of the cases involved in the complaint and tactical response report. For example, is a specific race more likely to become the victim of police officers’ use of force? Does the race of the police officer play a role in this as well? Whether cross-race interaction increases the likelihood of use of force.

Under each race, we will look into the environmental factors that could work together to influence the decisions of the force usage, including weather, physical location, time of the day. For example, whether the use of force occurs more often in the neighborhoods with a cluster of certain races and whether the reduced visibility of black people increases the incentive for force usage.



\section{Relational Analytics}

\begin{enumerate}

\item Information about victims and officers

    \begin{enumerate}

    \item What is the racial distribution of the victims involved in cases of use of force?

    \item What is the racial distribution of police officers involved in these cases?

    \item What portion of the total use of force cases involves an officer that is of a different race than that of the victim (cross-race use of force)?

    \item What portion of the cases in use of force cases contained firearm usage.

    \item What are the percentages of use of force cases grouped by officer race and subject race? (i.e. what is the percentage of white officers using force on black subjects)

    \end{enumerate}

\item Environmental factors that may affect an officer's decision to use force

    \begin{enumerate}

    \item What portion of the use of force happened under different lighting conditions?

    \item What portion of the use of force happened indoors against outdoors?

    \item What portion of the use of force happened under different weather conditions?

    \item What portion of the use of force happened under different locations?

    \item Under what combinations of different conditions (lighting, indoor or outdoor, weather, location) is a police officer more likely to use force?

    \end{enumerate}

\item How does race affect an officer's decision to use force in under different environment

    \begin{enumerate}

    \item How does the effect of lighting conditions vary from race to race?

    \item How does the effect of the indoor or outdoor scenario vary from race to race?

    \item How does the effect of the weather conditions vary from race to race?

    \item How does the effect of the locations vary from race to race?

    \item How does the top 10 combination of conditions vary from race to race?

    \end{enumerate}

\end{enumerate}





\section{Data Visualization}

\begin{enumerate}

\item Five histograms for top 5 conditions, each shows composition of the condition when use of force occurred to the four races (x-axis)

\item A box plot that lays out the age distribution of the victims for four races

\end{enumerate}



\section{Interactive Data Visualization}

\begin{enumerate}

\item Geo-spatial graphs of Chicago racial clusters for 4 major race categories (white, asian, hispanic, black) in each community/neighborhood. For each of these graphs, the use of force allegations would be layered on top to demonstrate the correlation between race composition and frequency of use of force.

\item A pie chart with the distribution of the race of the victims with the racial composition of the police officers involved packed in each pie section

\end{enumerate}




\section{Graph Analytics}

\begin{enumerate}

\item Do the use of force allegations happen in specific locations (with factors such as racial distribution, income distribution, etc...), or are they more spread out throughout the entire Chicago metropolitan area?

\item Are there clusters of police officers that are co-accused of use of force against victims of a different race?

\end{enumerate}



\section{Natural Language Processing}

Most of the complaint information is lost during classification, leading to preprocessed data that is already biased. We would like to further investigate the effect of race by digging deeper into the complaint reports. Specifically, we would like to see whether racial discrimination against non-white subjects plays a significant role in whether the police officer uses force or not.

\begin{enumerate}

\item Whether violent uses of force (i.e. ``... punched ... on the nose ...") happens more often to victim of non-white races?

    \begin{enumerate}

    \item Furthermore, investigate whether this is common for situations where the police officer and the subject are of different race?

    \end{enumerate}

\item Does the usage of vulgar language (``fucking ...") or racial slurs by the accused officer (``nigger") happen more often to victims of non-white races? What about cases where the police officer and the subject are of different races?

\end{enumerate}


\end{document}
