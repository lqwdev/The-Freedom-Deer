\documentclass[10pt]{article}
\usepackage[document]{ragged2e}
\usepackage{multicol}
\usepackage[margin=1in]{geometry}
\usepackage{titlesec}
\usepackage{fancyhdr}

\pagestyle{fancy}
\fancyhf{}
\fancyfoot[R]{P. \thepage}
\fancypagestyle{plain}{
    \renewcommand{\headrulewidth}{0pt}
    \fancyhf{}
    \fancyfoot[R]{P. \thepage}
}

\setlength{\parindent}{0em}
\setlength{\parskip}{1em}
\titlespacing*{\section}
{0pt}{0.5em}{0.5em}

\title{Project Proposal}
\author{The Freedom Deer: Tianchang Li, Hualiang Qin, Qingwei Lan}

\begin{document}
\maketitle


\section{Theme}

The theme of our research revolves around answering the following guiding questions:
\begin{itemize}
\setlength\itemsep{0em}
\item What makes someone more risky towards experiencing the use of force?
\item What are the typical scenarios in which police officers tend to use force?
\end{itemize}

Our main goal is to investigate the conditions under which police officers tend to use force on civilians. We believe that there are many reasons including the race of the subject, the environment under which the conflicts happened, the physical location in Chicago (neighborhood demographics), etc. We would like to explore how each of these conditions influences the police officer’s choice to use force and how they interact with each other to impact as a whole.

Specifically, we revolve our research on the differentiated treatments civilians receive from the police officers by studying the characteristics of the cases involved in the complaint and tactical response report. For example, is a specific race more likely to become the victim of police officers’ use of force? Does the race of the police officer play a role in this as well? Whether cross-race interaction increases the likelihood of use of force.

Under each race, we will look into the environmental factors that could work together to influence the decisions of the force usage, including weather, physical location, time of the day. For example, whether the use of force occurs more often in the neighborhoods with a cluster of certain races and whether the reduced visibility of black people increases the incentive for force usage.



\section{Relational Analytics}

\begin{enumerate}

\item Information about victims and officers

    \begin{enumerate}

    \item What is the difference between the subject race distribution and its distribution in the total population?

    \item What portion of these use of force cases involves an officer that is of a different race than that of the victim (cross-race use of force) and what are the racial distributions of the subjects and officers in each of these cases?

    \item What portion of use of force cases in tactical response reports involved police officer firearm usage?

    \end{enumerate}

\item Environmental factors that may affect an officer's decision to use force

    \begin{enumerate}

    \item What portion of the use of force happened under different lighting conditions?

    \item What portion of the use of force happened under different weather conditions?

    \item Under what combinations of different conditions (lighting, indoor or outdoor, weather, location) is a police officer more likely to use force?

    \item How does the influence of the top 10 combinations of different conditions vary from race to race?

    \end{enumerate}

\end{enumerate}





\section{Data Visualization}

\begin{enumerate}

\item A box plot that lays out the age distribution of the victims for five races. Another box plot on the side that further divide each of the five race groups into male and female

\item Three stacked bar charts, each illustrates the composition of lighting conditions/ weather conditions/ locations under which TRR cases occurred. In each bar chart, cases are further divided into five races, so that we can compare the “favored” conditions under which police officers tend to exercise force on each race.

\item A bubble plot that illustrates the percentage of each police-victim race pair: white-black, hispanic-black, white hispanic, etc.

\item A bar chart that compares the percentage of each race population in the whole population and the percentage of TRR cases occurred to each race in all TRR cases

\end{enumerate}



\section{Interactive Data Visualization}

\begin{enumerate}

\item Bubble chart view of cross race situation in period of 2004-2006, 2007-2009, 2010-2012, 2013-2016

\item Bar chart view of the use of force cases under four environmental factor (Lighting condition, Indoor/Outdoor, Weather, Location).

\item Sequence Sunburst Portraiting the Demographics of the Parties Involved in Use of Force.

\end{enumerate}




\section{Graph Analytics}

\begin{enumerate}

\item Two officers graphs
    \begin{enumerate}
    \item 1. Connect officers that have the same assignment shift time (baseline)
    \item 2. Connect officers that participate in the same trr event
    \end{enumerate}
\item Analysis of the triangle count and pagerank of these two graphs

\item By comparing the triangle counts of graph 1 and graph 2, analyze the anomalies of officers and evaluate their influence.

\end{enumerate}



\section{Natural Language Processing}

We want to see if the race of the subject and the heat in the situation have influence on the action of the use of force

\begin{enumerate}

\item We will obtain the "heat in situation" by applying a pretrained NLP model to calculate the sentimental score of the summary of the allegation. Also, we categorize allegation by the race of the subject/officer. Finally, we will run a regression model to see whether "Use of Force" will be influenced by these two factors and to what extent.

\end{enumerate}


\end{document}
